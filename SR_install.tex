

{\color{coolSection}\section{First steps}}

{\color{coolSubSection}\subsection{System requirement}}

\begin{description}
\item [Memory:] At least 1024 MB of RAM for a good functioning
\item [Java:] SmartRoot works with Java 1.5 or higher
\item [Database (optional):] SmartRoot can export data to .csv text files but also directly into a database. For Windows we present how to setup a MS Access database connection. For Mac OS X and Linux (Ubuntu), we present how to setup a MySQL connection. People who does not have MS Access on Windows can follows similar steps as for Linux and Mac OS to install a MySQL database.
\end{description}

%%%%%%%%%%%%%%%%%%%%%%

{\color{coolSubSection}\subsection{Installation files}}

\noindent Inside the SmartRootSetup.zip file, you will find the following folders and files:\\

\begin{description}
\item [SmartRoot Quick Start.pdf] This document. Helps you to quickly install SmartRoot.
\item[SmartRoot User Guide.pdf] Complete user guide to learn all the SmartRoot functionalities. 
\item [Quick Start Images folder] Four images to learn how to trace root with SmartRoot. Instructions are written directly on the images.
\item [SmartRoot folder] The SmartRoot program in itself. Contains four .jar files: Smart\_Root.jar, Image\_Explorer.jar, jcommon-1.0.16.jar, jfreechart-1.0.13.jar and mysql-connector-java-5.1.7-bin.jar. This is the folder you will have to copy in the ImageJ folder (see below).
\end{description}

%%%%%%%%%%%%%%%%%%%%%%

\newpage
{\color{coolSection}\section{Windows installation}}

\begin{enumerate}
\item Configure the database (optional)
\item Install ImageJ and SmartRoot
\item Use the Quick start images
\end{enumerate}

%{\color{coolSubSection}\subsection{Java 1.5 installation}}
%
%Download Java 1.5 from the following page (choose the Java Runtime Environment (JRE) 5.0 Update 22) and install it.\\
%
%\noindent \fbox{\parbox{\linewidth}{\url{http://java.sun.com/javase/downloads/index_jdk5.jsp}}}\\ 

{\color{coolSubSection}\subsection{Database installation and connection}}
\label{dbwin}

\noindent To configure an ODBC data source that connects SmartRoot to a MS Access database:\\

\begin{enumerate}
\item Close the ImageJ program if it is running
\item Starts the \verb|ODBC administrator| from the \\ \verb|Control panel > Administrative tools > ODBC administrator|
\item Under the tab \verb|User DSN|, click \verb|Add…|
\item A list of database drivers is displayed. Select \verb|Microsoft Access Driver|, and click \verb|Finish|. You may need to contact your DB vendor if the driver is not in that list.
\item In the next dialog box, specify \verb|SmartRoot| in the \textbf{Data Source Name} field. In the \textbf{Database} area, click \verb|Create| to create a new database. Choose the directory in which you want to create the database, and name it \verb|SmartRoot.mdb| (in the upper left text field).
\item Click OK to validate and quit the ODBC administrator.\\
\end{enumerate}

When you launch SmartRoot (see below), the following message is displayed in the Results window of ImageJ if the connexion was successfully established:

\begin{Verbatim}[frame=single, commandchars=+\(\)]
SQL connection started on ODBC source SmartRoot
\end{Verbatim}

If the program failed to open the datasource, the message is:

\begin{Verbatim}[frame=single, commandchars=+\(\)]
The ODBC datasource 'SmartRoot' was not found.
You will not be able to write to a database.
\end{Verbatim}

{\color{coolSubSection}\subsection{SmartRoot installation\\}}

\begin{enumerate}
\item Download and install ImageJ 
\item Copy the SmartRoot folder in the \verb|Program Files > ImageJ > Plugins| folder.
\item Open ImageJ and choose \verb|Plugins > SmartRoot > SR Explorer|\\
\end{enumerate}

\noindent ImageJ download:\\

\noindent\fbox{\parbox{\linewidth}{\url{http://rsbweb.nih.gov/ij/download.html}\\
\footnotesize{If you do not have Java installed, please choose a version of ImageJ bundled with Java}}}\\\\

\noindent
\fbox{\parbox{\linewidth}{%
{\color{red}\textsc{Important:}}\\
If you are using {\color{coolSection} Windows 7}, all the components you are using together (in our case, Java, ImageJ and Access) have to be build on the same architecture (32bit or 64bit). \\
\noindent For instance,  if your Access software is 64bit, please choose the {\color{coolSubSection}ImageJ bundled with 64 bit Java} in the ImageJ download page. 
}}\\

%%%%%%%%%%%%%%%%%%%%%%%%%%%%%

\newpage

{\color{coolSection}\section{Mac OSX installation}}

\begin{enumerate}
\item Install and configure MySQL database (optional)
\item Install ImageJ and SmartRoot
\item Use the Quick start images
\end{enumerate}

{\color{coolSubSection}\subsection{MySQL installation and configuration}}
\label{dbmac}

\subsubsection{Installation}
Download the latest MySQL version from:\\ 

\noindent \fbox{\parbox{\linewidth}{\url{http://dev.mysql.com/downloads/}}}\\

\noindent
Open the disk image then install MySQL by double clicking on the \verb|mysql-...-.pkg| icon. \\
Also install the \verb|MySQLStratupItem.pkg| and \verb|MySQL.prefPane|.\\

\noindent
Open the \verb|System Preferences>MySQL| and start the MySQL server.

\subsubsection{Configuration}

Download the MySQLWorkbench from the following link and install it \\

\noindent \fbox{\parbox{\linewidth}{\url{http://dev.mysql.com/downloads/workbench}}}\\

\noindent 
Open the application and click \verb|New Connection|. Fill the fields as follow:\\

\noindent
\fbox{ \parbox{\linewidth}{%
\textbf{Connection Name:} choose the name you want (ex: SmartRoot) \\ 
\textbf{Connection Method:} Standart (TCP/IP)\\
\textbf{Hostname:} localhost \\ 
\textbf{Port:} 3306 \\ 
\textbf{Username:} choose the name you want (ex: root) \\ 
\textbf{Password:} leave it empty \\ 
\textbf{Default Schema:} leave it empty }} \\

\noindent
Open the connection and create a new schema called \verb|SmartRoot| by clicking the '+' sign.\\
Name the new schema SmartRoot and create it. \\
Click the \verb|Refresh| button to see your newly created database.


{\color{coolSubSection}\subsection{SmartRoot installation \\}}

\begin{enumerate}
\item Download and install ImageJ 
\item Copy the SmartRoot folder in the \verb|Applications > ImageJ > Plugins| folder.
\item Open ImageJ and choose \verb|Plugins > SmartRoot > SR Explorer|\\
\end{enumerate}

\noindent ImageJ download:\\

\noindent\fbox{\parbox{\linewidth}{\url{http://rsbweb.nih.gov/ij/download.html}}}

\newpage
{\color{coolSubSection}\subsection{Connect SmartRoot to the database}}

Once you have installed SmartRoot, open it.
The following message is displayed in the Results window of ImageJ if the connexion was successfully established:\\

\begin{Verbatim}[frame=single, commandchars=+\(\)]
SQL connection started 
\end{Verbatim}

\noindent If the program failed to open the datasource, the message is:

\begin{Verbatim}[frame=single, commandchars=+\(\)]
The specified datasource was not found.
You will not be able to write to a database.
\end{Verbatim}

\noindent
If you see this error message, go in the SmartRoot window, choose the \verb|Settings| tab and find the \verb|SQL options| panel. Fill the fields as follow:\\

\noindent
\fbox{\parbox{\linewidth}{%
\textbf{Driver class name:} com.mysql.jdbc.Driver\\
\textbf{Connection URL:} jdbc:mysql://localhost/SmartRoot\\
\textbf{Connection user name:} the username you choose previously\\
\textbf{Connection password:} leave empty }}\\
	
\noindent Press the \verb|Save Prefs| then \verb|Restart server| button. You should see the correct message saying the connection started


%%%%%%%%%%%%%%%%%%%%%%%%%%%%%%%%%%%%
\newpage
{\color{coolSection}\section[Linux installation]{Linux installation (Ubuntu distribution)}}

\begin{enumerate}
\item Install MySQL (optional)
\item Install ImageJ
\item Install ImageJ and SmartRoot
\item Configuring the database (optional)
\item Use the Quick start images
\end{enumerate}

%{\color{coolSubSection}\subsection{Java 1.5 installation}}
%
%In the terminal window type:\\
%
%\begin{Verbatim}[frame=single, commandchars=+\(\)]
%$sudo apt-get install sun-java5-jre
%\end{Verbatim}


{\color{coolSubSection}\subsection{MySQL installation and configuration}}
\label{dblin}

\subsubsection{Installation}

In the terminal window type:\\

\begin{Verbatim}[frame=single, commandchars=+\(\)]
$sudo apt-get install mysql-server
$sudo apt-get install mysql-query-browser
\end{Verbatim}

\noindent
While installing, you will be asked to setup username and password for your database connection. Leave the default values.

\subsubsection{Configuration}

Open MySQL Administrator. To connect to the database fill the form as follow:\\

\noindent
\fbox{\parbox{\linewidth}{%
\textbf{Server Hostname:} localhost\\
\textbf{Username:} root\\
\textbf{Password:} Leave empty}}\\

\noindent
In the MySQL Administrator window, choose \verb|Catalog| in the left panel. 

\noindent
In the bottom left panel \verb|Schemata|, right-click and choose \verb|Create Schema|. \\
Name it \verb|SmartRoot|



{\color{coolSubSection}\subsection{ImageJ installation}}

\noindent
In the terminal window type:\\

\begin{Verbatim}[frame=single, commandchars=+\(\)]
$sudo apt-get install imagej
\end{Verbatim}



{\color{coolSubSection}\subsection{SmartRoot installation}}

Copy the \verb|SmartRoot| folder from the \verb|SmartRootSetup| folder you downloaded into the \verb|usr/share/imagej/plugins/| folder\\

\noindent
In the terminal window type:\\

\begin{footnotesize}
\begin{Verbatim}[frame=single, commandchars=+\(\)]
$sudo mv /home/(+color(red)where_you_unzipped)/SmartRootPlug/SmartRoot /usr/share/imagej/plugins
\end{Verbatim}
\end{footnotesize}

\noindent To launch SmartRoot open ImageJ and choose \verb|Plugins > SmartRoot > SR Explorer|\\

\noindent
\fbox{\parbox{\linewidth}{%
{\color{red}\textsc{Important:}}\\
Ubuntu use the {\color{coolSubSection}Alt-key} to grab and move windows. SmartRoot use the same key to automatically trace roots. In order to use SmartRoot correctly, you have to change one Ubuntu parameter:\\

\noindent Go to System $>$ Preferences $>$ Windows and set the Movement key to Super.}}\\


{\color{coolSubSection}\subsection{Connect SmartRoot to the database}}

Once SmartRoot is installed, open it.

\noindent
The following message is displayed in the Results window of ImageJ if the connexion was successfully established:\\

\begin{Verbatim}[frame=single, commandchars=+\(\)]
SQL connection started 
\end{Verbatim}

\noindent If the program failed to open the datasource, the message is:

\begin{Verbatim}[frame=single, commandchars=+\(\)]
The specified datasource was not found.
You will not be able to write to a database.
\end{Verbatim}

\noindent
If you see this error message, go in the SmartRoot window, choose the \verb|Settings| tab and find the \verb|SQL options| panel. Fill the fields as follow:\\

\noindent
\fbox{\parbox{\linewidth}{%
\textbf{Driver class name:} com.mysql.jdbc.Driver\\
\textbf{Connection URL:} jdbc:mysql://localhost/SmartRoot\\
\textbf{Connection user name:} the username you choose previously (default = root)\\
\textbf{Connection password:} leave empty }}\\
	
\noindent
Press the \verb|Save Prefs| then \verb|Restart server| button. You should see the correct message saying the connection started

%%%%%%%%%%%%%%%%%%%%%%%%%%%%%%%%%%%%%%%%%





